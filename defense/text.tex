% Created 2019-02-03 Sun 14:59
% Intended LaTeX compiler: pdflatex
\documentclass[11pt]{article}
\usepackage[utf8]{inputenc}
\usepackage[T1]{fontenc}
\usepackage{graphicx}
\usepackage{grffile}
\usepackage{longtable}
\usepackage{wrapfig}
\usepackage{rotating}
\usepackage[normalem]{ulem}
\usepackage{amsmath}
\usepackage{textcomp}
\usepackage{amssymb}
\usepackage{capt-of}
\usepackage{hyperref}
\usepackage{color}
\usepackage{minted}
\usepackage{color}
\usepackage{minted}
\usepackage{parskip}
\usepackage[options]{qcircuit}
\usepackage{geometry}
\geometry{left=2.5cm,right=2.5cm,top=2.5cm,bottom=2.5cm}
\author{Daniel Moreno Manzano}
\date{\today}
\title{Thesis Defense}
\hypersetup{
 pdfauthor={Daniel Moreno Manzano},
 pdftitle={Thesis Defense},
 pdfkeywords={},
 pdfsubject={},
 pdfcreator={Emacs 25.1.1 (Org mode 9.1.14)}, 
 pdflang={English}}
\begin{document}

\maketitle
\tableofcontents




\section{Intro to Quantum}
\label{sec:org47caaa5}

\subsection{Qubits and what is Quantum (\textasciitilde{}2 min)}
\label{sec:orga338936}

\subsubsection{Intro}
\label{sec:org84e281f}

First of all I would like to introduce a bit about what is quantum computing.
Throughout the last century, the discoveries in Quantum Physics gave a headache to all the physicists.
There were two phenomena which behaviour was far from what we could understand from classical physics: \textbf{superposition} and \textbf{entanglement}.

Still today there is no explanation to this phenomena but, Feynman -- one of those physicists -- noticed that this behaviour could be used to compute large problems much faster than classical computers.

\subsubsection{Qubit}
\label{sec:orgf7838a3}

So they made up the concept of quantum bit -- the qubit --, that is the basic unit of quantum information.
A qubit state is described in the bra-ket notation.
Due to the \textbf{superposition} effect and unlike a bit, a qubit could be either in the excited state, in the ground state or even \textbf{both} at the same time.
In other words, it can be 0 and 1 at the same time.

As an analogy, this is like flipping a coin.
While it is turning in the air one cannot distinguish either face or tails, but both.

[Superposition graph]

[coin animation]

The superposition phenomena is mathematically represented as this formula there

$${\displaystyle |\psi \rangle =\alpha_0 |0\rangle +\alpha_1 |1\rangle }$$


And, whenever the state of the qubit is measured, the measurement returns either 0 with a probability of  \(|\alpha_0| ^2\) or 1 with a probability of \(|\alpha_1| ^2\).

[graph representing the possibilities of measuring a qubit]

For instance, a qubit in the state \(|\psi \rangle =\sqrt{0.3} |0\rangle + \sqrt{0.7} |1\rangle\), will output 0 30\% of the times and 1 70\% percent of the times.

But, -- and here comes the strangest part -- after measurement the qubit collapses in the measured state.

Following the coin example, it is like, when you catch the coin, until you don't open the hands it could be in both states still.
But, whenever you look at it you make the coin stay in either heads or tails.

I'm sure you've heard about the famous Shrödinger's cat.
It is the same. 
The cat is alive and dead at the same time until we open the box and we wither kill or maintain him alive.

[A cat somewhere?]

\subsection{Motivation. Why quantum is so cool?}
\label{sec:org4da5130}

Ok, but why everybody is so excited about quantum?

Well as I said before, quantum computers have much more computational power than classical computers.
Due to the superposition effect, with \(N\) qubits we can represent \(2^{2N}\) states, while with bits we can represent \(2^N\).

\begin{center}
\begin{tabular}{ll}
\(N\) bits & \(N\) qubits\\
\(2^N\) & \(2^{2N}\)\\
\end{tabular}
\end{center}

For instance, with 3 qubits we can represent either 000, 001, etc. Till 111; 8 states.
But with 3 qubits we can represent all the those states at once.

\begin{center}
\begin{tabular}{ll}
3 bits & 3 qubits\\
gif of the possibilities & \(\alpha_{000} \textbar 000\rangle +\alpha_{001} \textbar 001\rangle + ... + \alpha_{111} \textbar 111\rangle\)\\
\end{tabular}
\end{center}


[Graph showing the relation between qubit states and bit states]

[more quantum promising things?]


\subsection{Quantum Gates and circuits}
\label{sec:orgcdfc933}

In order to do operations with qubits the quantum gates are used.
They are the quantum siblings of the boolean gates.
As described in the report, depending on the gate you apply to a qubit you will change it's state in different ways.
There are two main kinds, single-qubit gates and two-qubit gates.

[Boolean gates figure]

These are the single-qubit gates:

\begin{center}
\begin{tabular}{lllllll}
Identity & Pauli-X & Pauli-Y & Pauli-Z & Hadamard & S gate & T gate\\
 &  &  &  &  &  & \\
\Qcircuit @C=1em @R=.7em {
  \lstick{|q\rangle} & \gate{I} & \qw\\
}
 & \Qcircuit @C=1em @R=.7em {
  \lstick{\shortmid q\rangle} & \gate{I} & \qw\\
}
 & \Qcircuit @C=1em @R=.7em {
  & \gate{Y} & \qw\\
}
 & \Qcircuit @C=1em @R=.7em {
  \lstick{|q\rangle} & \gate{Z} & \qw\\
}
 & \Qcircuit @C=1em @R=.7em {
  & \gate{H} & \qw\\
}
 & \Qcircuit @C=1em @R=.7em {
  \lstick{|q\rangle} & \gate{S} & \qw\\
}
 & \Qcircuit @C=1em @R=.7em {
  \lstick{|q\rangle} & \gate{T} & \qw\\
}
\\
\end{tabular}
\end{center}


And these three, the CNOT, the CZ and the SWAP are the two-qubit gates.
For example, a CNOT is a Controlled-NOT gate.
The qubit in the black dot is the control qubit, and the one going through the plus symbol is the target qubit.
Depending on the state of the control qubit it will negate, or not, the target one.

An easier example is the SWAP gate, that exchanges the states of two qubit.
Moving the state of one qubit to the other and vice versa.

\begin{center}
\begin{tabular}{lll}
CNOT & CZ & SWAP\\
 &  & \\
\Qcircuit @C=1em @R=.7em {
  & \targ & \qw\\
  & \ctrl{-1} & \qw\\
}
 & \Qcircuit @C=1em @R=.9em {
  & \ctrl{1} & \qw\\
 & \control \qw & \qw\\
}
 & \Qcircuit @C=1em @R=.7em {
 & \qswap & \qw\\
  & \qswap \qwx[-1] & \qw\\
}
\\
\end{tabular}
\end{center}


\subsubsection{Quantum Circuits}
\label{sec:org79a6fca}

As boolean logic, the quantum algorithms are represented as the so-called quantum circuits.
As algorithm representations, they are \textbf{hardware agnostic}.
Or in other words, quantum circuits do not consider the device where they are going to be executed.

A quantum circuit is based on \textbf{virtual qubits} -- because they are a representation, not the real ones in the device -- and quantum gates.
The quantum circuits connect the virtual qubits with the upcoming quantum gates that are going be applied to them.

For instance, in this circuit we can see 6 qubits connected with 5 CNOTs and 6 Measurements at the end.

\begin{figure}[H]
    \centering

\resizebox{0.3\textwidth}{!}{
   \Qcircuit @C=1em @R=.7em {
\lstick{q_1} & \targ & \qw & \qw & \qw & \qw & \push{|0\rangle} \qw & \meter & \rstick{0} \qw\\
\lstick{q_2} & \ctrl{-1} & \targ & \qw & \qw & \qw & \push{|0\rangle} \qw & \meter & \rstick{0} \qw\\
\lstick{q_3} & \qw & \ctrl{-1} & \targ & \qw & \qw & \push{|0\rangle} \qw & \meter & \rstick{0} \qw\\
\lstick{q_4} & \qw & \qw & \ctrl{-1} & \targ & \qw & \push{|0\rangle} \qw & \meter & \rstick{0} \qw\\
\lstick{q_5} & \qw & \qw & \qw & \ctrl{-1} & \targ & \push{|0\rangle} \qw & \meter & \rstick{0} \qw\\
\lstick{q_6} & \qw & \qw & \qw & \qw & \ctrl{-1} & \push{|1\rangle} \qw & \meter & \rstick{1} \qw
}
}
\end{figure}


\subsubsection{Qubits are faulty. Example of a circuit demonstrating that qubits are faulty?}
\label{sec:org2ddf9c0}

But \textbf{qubits and quantum gates are faulty}.
Qubits have a limited life time, called the \textbf{decoherence time}, and quantum gates can introduce errors.

Let's say that the correct answer of this circuit [previous example] is \texttt{000001}.
We can see how, under ideal conditions, after the gates the qubits hold a pure state that is correctly measured.

But, if we add errors, the results will be much more different.
We can see now in the circuit how the first and the last qubit are still correct, but the rest have been affected by errors.

\begin{figure}[H]
    \centering

\resizebox{0.3\textwidth}{!}{
   \Qcircuit @C=1em @R=.7em {
\lstick{q_1} & \targ & \qw & \qw & \qw & \qw & \push{|0\rangle} \qw & \meter & \rstick{0} \qw\\
\lstick{q_2} & \ctrl{-1} & \targ & \qw & \qw & \qw & \push{|0\rangle} \qw & \meter & \rstick{1} \qw\\
\lstick{q_3} & \qw & \ctrl{-1} & \targ & \qw & \qw & \push{\sqrt{0.3}|0\rangle + \sqrt{0.7}|1\rangle} \qw & \meter & \rstick{1} \qw\\
\lstick{q_4} & \qw & \qw & \ctrl{-1} & \targ & \qw & \push{\sqrt{0.3}|0\rangle + \sqrt{0.7}|1\rangle} \qw & \meter & \rstick{0} \qw\\
\lstick{q_5} & \qw & \qw & \qw & \ctrl{-1} & \targ & \push{\sqrt{0.3}|0\rangle + \sqrt{0.7}|1\rangle} \qw & \meter & \rstick{1} \qw\\
\lstick{q_6} & \qw & \qw & \qw & \qw & \ctrl{-1} & \push{|1\rangle} \qw & \meter & \rstick{1} \qw
}
}
\end{figure}

We can see how the second qubit is correct until the measurement and then it suffers a measurement error.
We can see, as well, that qubits 3, 4 and 5 suffer an error during the circuit that leads them to an incorrect state.
As explained before, this error makes the qubit to collapse in 0 30\% of the times and in 1 70\% of the times.

\subsection{Quantum is real}
\label{sec:org6aad16b}

All right, this was theory.
Let's talk about reality.

\subsubsection{Quantum chip examples}
\label{sec:orge9d45c8}

Quantum Computing is not just a theoretical field.
Quantum Computers do exist!
For example, QuTech's Superconducting chips, SC-7 and SC-17.

We can see the golden structure and the fridge required to set the proper conditions for the quantum chip.
Normally, quantum chips need to be at a very cold temperature, \textasciitilde{}2°K.
Moreover, we can see in the picture how the controllers -- that are at room temperature -- and the chips are separated.

[Figure of the fridge and the golden structure]

In this research, we will focus our work in the SC-17 chip.

\subsubsection{Chip constrains \(\to\) Why mapping?}
\label{sec:org639bdbe}

Quantum chips have constrains.
Most of the quantum chips -- not only QuTech's ones -- have one main limitation, the Nearest Neighbor constrain.
This constrain limits the interaction between qubits.

[Figure of the chip layout]

As one can see in the SC-17 chip layout, the qubits -- the colored dots -- are not connected in an all-to-all fashion, but in a grid fashion -- each qubit has a maximum of 4 connections.
For instance, we can see how qubit 8 is linked with qubits 5, 6, 10 and 11 but not with 7 or 9.

Therefore, if we want to run any quantum algorithm -- that is hardware agnostic -- in a given device you need to adapt/transform it to the requirements of the chip.
This adaption process is what we call the mapping procedure.

\section{Mapping}
\label{sec:org04d1802}
\subsection{What is mapping}
\label{sec:orgc4f2b96}

\ldots{}

\begin{itemize}
\item Scheduler
\item Initial placement
\item Router
\end{itemize}


\subsubsection{Scheduler}
\label{sec:orge24e8c4}

\begin{enumerate}
\item Definition + Dependence graph
\label{sec:orge8c7094}

\item Example
\label{sec:org296ad38}
\end{enumerate}
\subsubsection{Initial placement}
\label{sec:orgfb46bf6}
\begin{enumerate}
\item Definition
\label{sec:orgbaa068c}
\end{enumerate}
\subsubsection{Router}
\label{sec:org60943ef}
\begin{enumerate}
\item Definition
\label{sec:org9c82baf}
\end{enumerate}
\subsubsection{Metrics intro?}
\label{sec:orgaf8f759}
\subsection{Example}
\label{sec:orgbba9723}
\subsection{Metrics outro}
\label{sec:org3136856}
\section{Constrains?}
\label{sec:org5bf8fbe}
\subsection{More constrains}
\label{sec:org1320c1c}

[Figure of SC-17 again]

Particularly, SC-17, as most of the superconducting technologies, has more constrains that we need to take into account.
As I describe with more detail in the thesis:

\begin{itemize}
\item SC-17 does not support all quantum gates, but enough to do any possible operation.
\item SC-17 requires between 20 and 600 \(ns\) to execute the quantum gates, depending on their kind. As explained before, time is a critical issue in quantum, so the faster we run the better.
\item SC-17 has also a limitation in terms of parallel execution of gates. Not all single- or two-qubit gates are allowed to be executed at the same time.
\item Finally, SC-17 has three feed-lines able to measure out the qubits' state. As it can be seen in the SC-17 figure, each feed-line is connected to a different set of qubits; and whenever one uses a feed-line it will measure all the qubits. So, although one can extract the information of only one qubit, the rest of qubits in the same feed-line will be measured as well. This makes impossible to measure two qubits in the same feed-line in a row. For example, we cannot measure qubit 2 and then qubit 6.
\end{itemize}

\section{Aim of the project again}
\label{sec:org152a912}
\subsection{Aim}
\label{sec:orgbd3c714}
\subsection{Overall view of what we have done}
\label{sec:org26b7d2a}

\begin{itemize}
\item New Metrics
\item Benchmarks
\item Framework Analysis (Map + simulation)
\end{itemize}

\section{New metrics}
\label{sec:org6562b5c}

\subsection{Fidelity}
\label{sec:org34c145d}

\subsection{Probability of success}
\label{sec:org9884af3}

\subsection{Quantum Volume}
\label{sec:org516ee7c}
\section{Benchmarks}
\label{sec:orgbf1375d}
\subsection{Why Benchmarks}
\label{sec:orgf4b0049}
\subsection{How Benchmarks (sources)}
\label{sec:org3b51801}
\subsection{How the Benchmarks are}
\label{sec:orgf574c69}
\subsection{Selection}
\label{sec:orgda14697}
\section{Framework Analysis}
\label{sec:orgf7efb16}
\subsection{OpenQL}
\label{sec:org1bb70bc}
\subsection{quantumsim}
\label{sec:org71ad76b}
\subsection{Overall view of the Framework analysis}
\label{sec:org48e2ac6}
\section{Results}
\label{sec:org1c42612}
\subsection{Impact of mapping}
\label{sec:org8bf4a01}
\subsubsection{Fidelity decreases}
\label{sec:org21f5871}
\subsubsection{How fidelity decreases?}
\label{sec:org0a7e0f2}
\subsection{Analysis of the metrics}
\label{sec:org07623f2}
\subsubsection{Probability of success and fidelity correlation}
\label{sec:orge9f158a}
\subsubsection{Metrics correlation}
\label{sec:orgee19340}
\subsection{Future Work}
\label{sec:org17c2287}
\end{document}
