

\chapter{Conclusions and Future Work}
\label{sec:org3d18185}
\section{Conclusions}
\label{sec:org47e21c4}
\section{Future work}
\label{sec:org6667f25}

While reviewing our results we recognized several issues that
These topics are deferred to future work.

\subsection{More samples}
\label{sec:orgf9d20fa}

Although our results explore a big set of values for each metric, we 


\subsection{More qubits}
\label{sec:org0310707}

Also, we targeted 

\subsection{Improve Quantum Volume}
\label{sec:orgc127c02}

As we mention in the \hyperref[]{Quantum Volume} section, the metric of Quantum Volume defined as the multiplication of qubits and depth of the circuit is an initial attempt of studying it as mapping metric.
As we showed in the \hyperref[]{Results} chapter, this definition, although highly correlated with the error increase, it is not having the best results.
Thus, further studies, which take a better definition of Quantum Volume, will need to be undertaken.

\subsection{Fidelity and probability of success estimation}
\label{sec:orgc5a25c3}

One of the main restrictions we have found through the development of this work are the simulation limits.
Simulations of long circuits require long time to be simulated and, in some cases, are not even able to be simulated.
In order to avoid this overload that drags out our work we would like to model fidelity and probability of success given the different circuit parameters.
Further studies on different and more samples configurations are therefore required in order to elucidate the regression model able to estimate fidelity and probability of success with the least error.

\subsection{Understanding more about probability of success}
\label{sec:org37f3ade}

This research has raised many questions in need of further investigation.





\subsection{Improve mapping algorithm}
\label{sec:org36e47c9}

Finally, aside from the aim of this project, we noticed that, in general, our mapping algorithm did not have state-of-the-art results.
Although it helped us to observe a higher range of values, the mapping algorithm should be able to not accumulate the amount of errors we show.
Research into solving this problem is already underway.
