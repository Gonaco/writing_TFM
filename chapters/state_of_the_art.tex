% Created 2018-12-13 Thu 02:57
% Intended LaTeX compiler: pdflatex
\documentclass[11pt]{article}
\usepackage[utf8]{inputenc}
\usepackage[T1]{fontenc}
\usepackage{graphicx}
\usepackage{grffile}
\usepackage{longtable}
\usepackage{wrapfig}
\usepackage{rotating}
\usepackage[normalem]{ulem}
\usepackage{amsmath}
\usepackage{textcomp}
\usepackage{amssymb}
\usepackage{capt-of}
\usepackage{hyperref}
\usepackage{color}
\usepackage{minted}
\usepackage{color}
\usepackage{minted}
\usepackage{parskip}
\author{Daniel Moreno Manzano}
\date{\today}
\title{}
\hypersetup{
 pdfauthor={Daniel Moreno Manzano},
 pdftitle={},
 pdfkeywords={},
 pdfsubject={},
 pdfcreator={Emacs 25.1.1 (Org mode 9.1.14)}, 
 pdflang={English}}
\begin{document}

Quantum algorithms are meant to leverage the promising power of quantum computers \cite{coles18:quant_algor_implem_begin}.
Commonly described as quantum circuits, quantum algorithms are hardware agnostic \cite{Nielsen_2009}.
Due to the variety of technologies that emerged to build quantum memories, the algorithms tend to be as theoretical as possible.
From ion traps [? paper on ion traps] to superconducting qubit \cite{Barends_2014}, through quantum dots \cite{Hill_2015,Li_2018}, each layout has its own requirements and constraints.
Also, the qubit chip layouts from IBM \cite{IBM_QX}, Google \cite{boixo16:charac_quant_suprem_near_term_devic} and Rigetti \cite{Sete_2016} are quite limited.

Moreover, quantum devices -- no matter which technology -- are error prone.
Quantum operations are faulty and qubits are not able to hold the desired state for long times, gradually rotating to another state -- the qubit decoheres.
\uline{[some numbers for the technologies]} \cite{O_Brien_2017}.
This creates an undesirable environment to compute the most useful algorithms.
Therefore, in order to fight the errors generated by this behaviour, fault-tolerant (FT) and quantum error correction (QEC) mechanisms have been developed during the last years \cite{Nielsen_2009} [? papers on error correction].
These techniques force the quantum chips layout to arrange the qubits in a particular manner \cite{Versluis_2017}, constraining them even more.

Thus, a link between the algorithms and the devices is required \cite{Fu_2016}.
As in classical computation, the algorithms should go through a compilation process in order to adapt them to the hosting device.
Certainly, the mapping procedure is an important part of this process based on three sub-tasks: scheduling, initial placement and routing; as we considered before.

There is a considerable amount of literature on the mapping task.
Initial works on this field \cite{Metodi_2006,Whitney_2007,Bahreini_2015} focused primarily on the definition of what they characterizedp a \emph{scheduler} able to parallelize operations and add the require ones to route qubits.
They would consider general constraints, common for most of the hardware devices -- although the works were examining ion-traps as hardware implementations.
The proposed techniques examine a dependency graph looking for the best way to organize qubits and operations.
The majority of the methods use latency as the metric to minimize, however some of them \cite{Farghadan_2017} would minimize in number of SWAP operations.
Following a similar reasoning as the first approaches, more complex solutions \cite{booth18:compar_integ_const_progr_tempor} have been published.
Also, several publications \cite{Lye_2015,Wille_2016} outlining only the routing sub-task using the number of SWAPS as the metric to minimize.
A recent review of the literature on the mapping topic \cite{zulehner17:effic_method_mappin_quant_circuit,Siraichi_2018,mckay18:qiskit_backen_specif_openq_openp_exper,Dueck_2018,Venturelli_2018} focused on device specific mapping algorithms, with promising results.

Many attempts have been made \cite{Dousti_2014,Heckey_2015,hwang18:hierar_system_mappin_large_scale,murphy18:contr,Lao_2018} with the purpose of develop a FT mapping able to work at the logical -- qubit -- level.
However, due to the high complexity of the QEC techniques, quantum chips with large amounts of qubits are still theory.
More recent evidence \cite{Preskill_2018}, proposes the Noisy Intermediate-Scale Quantum (NISQ) devices as the next step for near future hardware with an amount of 50-100 qubits and without QEC or much simpler encodings.
Several studies, for instance \cite{tannu18:case_variab_aware_polic_nisq,paler18:nisq,paler18:influen_initial_qubit_placem_durin}, have been conducted on the mapping algorithms required for NISQ devices.


\bibliography{../../thesis_plan}
\bibliographystyle{plain}
\end{document}
